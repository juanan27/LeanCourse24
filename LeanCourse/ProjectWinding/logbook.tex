\documentclass[a4paper,12pt]{article}

% Packages
\usepackage[utf8]{inputenc}    % Input encoding
\usepackage[T1]{fontenc}       % Font encoding
\usepackage{amsmath}           % Math symbols
\usepackage{amsfonts}          % Math fonts
\usepackage{amssymb}           % Extra symbols
\usepackage{geometry}          % Page geometry
\geometry{a4paper, margin=1in} % Set page size and margins
\usepackage{graphicx}          % Include graphics
\usepackage{hyperref}          % Hyperlinks in the document

% Title and Author Information
\title{Winding Number of a Curve - LEAN4 Project (Logbook)}
\author{Jorge Carrasco Coquillat \and
Juan Antonio Montalbán Vidal}
\date{\today} % Use \date{} to leave the date empty or set a custom date

\begin{document}

% Title Page
\maketitle

% Abstract
\begin{abstract}
In this logbook we aim to show how the journey with this project has been since day 0.
The content is divided in weeks, so we explain what we have been doing weekly,
decisions made and main difficulties we have had to tackle.
\end{abstract}

% Sections
\section{Choosing our project topic}
There were various options that seemed good for us, from Carmichael numbers
to orientability of manifolds. Lastly, we decided to work in a complex analysis topic:
the winding number.

The winding number of a curve can be defined in several ways, although we only
focused on the topological and analytic ones. Our main goal was going to be
to prove the equivalence between these definitions, albeit we quickly noticed
it was going to be a very stretch goal due to its complexity and the time we had.

Since this project focuses on the definition and properties of the winding number,
we will now give the definition of \textit{curve} that we have used - there is no a general
consensus on how to define them, especially regarding the definition interval.

For us, a curve $\gamma$ will be a $\cal{C}^1 (I, \mathbb{C})$ function, where I stands for the unit interval.

\begin{itemize}
  \item \textbf{Topological definition:} it uses the path-lifting property [ref]. Given
  a curve $\gamma$
  \item \textbf{Analytic definitions:} seguir
\end{itemize}

\section{Main difficulties}
Here is the main content of the article.

\subsection{Subsection Example}
This is a subsection.

\section{Conclusion and possible future work}
This is the conclusion of the article.

% References
\begin{thebibliography}{9}
\bibitem{sample1} Author Name, \textit{Book Title}, Publisher, Year.
\bibitem{sample2} Another Author, \textit{Another Book}, Another Publisher, Year.
\end{thebibliography}

\end{document}
