\documentclass[a4paper,12pt]{article}

% Packages
\usepackage[utf8]{inputenc}    % Input encoding
\usepackage[T1]{fontenc}       % Font encoding
\usepackage{amsmath}           % Math symbols
\usepackage{amsfonts}          % Math fonts
\usepackage{amssymb}           % Extra symbols
\usepackage{geometry}          % Page geometry
\geometry{a4paper, margin=1in} % Set page size and margins
\usepackage{graphicx}          % Include graphics
\usepackage{hyperref}          % Hyperlinks in the document
\usepackage{amsthm}

\newcommand{\imm}{\operatorname{im}}
\newtheorem{theorem}{Theorem}
\newtheorem{lemma}{Lemma}
% Title and Author Information
\title{Winding Number of a Curve - LEAN4 Project}
\author{Jorge Carrasco Coquillat \and
Juan Antonio Montalbán Vidal}
\date{\today} % Use \date{} to leave the date empty or set a custom date

\begin{document}

% Title Page
\maketitle

% Abstract
\begin{abstract}
In this document we aim to share our experience with this project, including the first problems we encountered,
such as choosing the project topic, main results we have formalized alongside with the difficulties we have had to
overcome. We finish with some conclusions and a short reflection. We have mainly followed \cite{COB22}.
\end{abstract}

% Sections
\section{Choosing our project topic}
There were various options that seemed good for us, from Carmichael numbers
to orientability of manifolds. Lastly, we decided to work in a complex analysis topic:
winding numbers.

The winding number of a curve can be defined in several ways, although we only
aimed towards the topological and analytic definitions. We firstly thought about
proving the equivalence between these definitions, albeit we quickly noticed
it was going to be too difficult due to its complexity and the time we had. 
This is why we decided to focus on the definitions and properties of winding numbers.

We will now give the definition of \textit{curve} that we have used - there is not a general
consensus on how to define them in LEAN, especially regarding the definition interval.

For us, a curve $\gamma$ will be a $\cal{C}^1 (\text{I}, \mathbb{C})$ function, where I stands for the unit interval.

\begin{itemize}
  \item \textbf{Topological definition:} it uses the path-lifting property \cite{sample2}. Given
  a curve $\gamma : [0, 1] \to \mathbb{C}$ and $z_0 \in \mathbb{C}$ satisfying $z \notin \text{im} \gamma$, the winding
  number of $\gamma$ around $z_0$ is defined as
  $$\omega (\gamma, z_0) = s(1) - s(0),$$
  where $(\rho, s)$ is the path written in polar coordinates, this is, the lifted path
  through the covering map
  $$p : \mathbb{R}_+ \times \mathbb{R} \to \mathbb{C}$$
  $$(\rho_0, s_0) \mapsto z_0 + \rho_0 e^{i2\pi s_0}.$$
  \item \textbf{Analytic definition:} The analytic definition is more straightforward. Consider the curve $\gamma$
  and the point $z_0$ as previously specified. The winding of $\gamma$ around $z_0$ is defined as
  $$\omega(\gamma, z_0) = \frac{1}{2\pi i} \int_{0}^{1} {\frac{\gamma'(t)}{\gamma(t)-z_0} \text{dt}}.$$
\end{itemize}

After spending half a week discussing whether we should try proving the equivalence of definitions we
noticed that there was a big problem regarding this issue: there are a lot of results missing in Mathlib
--- especially from algebraic topology --- which we would had need just to get to define the winding number.

\section{Main formalized results}

\noindent \textbf{Remark:} the sign of the winding number depends on the orientation we choose for going along
the curve or, in other words, if the curve winds \textit{clockwise} or \textit{counterclockwise} around the point.

The main results we have formalized are the following:
\begin{theorem}
    Given a curve $\gamma$ and a point $z_0$ satisfying $z \notin \text{im} \gamma$, then
    $$\omega(\gamma, z_0) \in \mathbb{Z}.$$
    In other words, the winding number of a curve around a point is always an integer.
\end{theorem}

\begin{theorem}
  Given a curve $\gamma$ with winding number $n \in \mathbb{Z}$, the winding number of the reverse curve
  is $-n$.
\end{theorem}

\begin{theorem}
  The winding number $\omega(\gamma, z)$, seen as a function of $z$, is continuous in $\mathbb{C}\setminus \imm \gamma$.
\end{theorem}

\begin{theorem}
Let $\gamma$ be the unit circle (parametrized in [0,1] by $t \mapsto e^{2\pi i t}$ . If $|z_0| > 1$, then 
$\omega(\gamma, z_0) = 0$. And, if $|z_0| < 1$, then $\omega (\gamma, z_0) = 1$.
\end{theorem}
\section{Main difficulties and solutions}
\begin{itemize}
  \item Coming up with the most suitable definition for curves. The very first thing we had to tackle, far from a lemma or theorem, was giving the right - this is, the
  most accurate one - definition of a curve. First we opted for defining it for an arbitrary interval $(a, b)$
  but we rapidly noticed this was unnecesary and the vast majority of the literature simply used the unit interval,
  so we decided to change to that.
  \item But that was not the unique thing we had to change about the structure definition of a (closed) curve. At the beginning, we opted
  for putting the continuity and differentiability conditions separately but, after hearing the given advice, we noticed
  the best option was to put them together using the \verb|ContDiffOn| condition from Mathlib library.
  \item Proving continuity of the index function. Mathlib misses results missing continuity of parametric
  integrals (which is a fundamental concept in mathematics, namely in analysis and PDEs) so we had to do it the classical
  way - using the $\varepsilon - \delta$ definition of continuity. It is usually tedious to work with this concept
  on paper, thus doing it in LEAN4 has been, sometimes, a big trouble.
  \item As an extra point, we would like to add how the harsh treatment some analytic expressions require. Further analysis and
  comment on this can be found in the conclusions.
\end{itemize}

\section{\textit{Sorrys} left to be solved in the project}
The only \textit{sorrys} that were left in the project are present in the document 'winding_continous.lean'. 
Most of them we think are not difficult to solve, but rather tedious and we did not have time to do so.
\section{Conclusion and possible future work}
As we discussed in the presentation, one of the main conclusions that arise from this project is that
Mathlib still needs some contributions - there exist several gaps regarding important results and concepts.
As well, refining the treatment and work with integrals would be a good improvement. For future work, the 
equivalence of the analytic definition of winding numbers given in this project and the topological definition
could also be proven.
% References
\begin{thebibliography}{9}
\bibitem{COB22} Cobos Díaz, Fernando, \textit{Notas de clase de Análisis Complejo}, 2022.
\bibitem{sample2} Wilkins, D. R. \textit{Course 212 (Topology), Academic Year 1991—92, Section 09: Winding Numbers}, \url{https://www.maths.tcd.ie/~dwilkins/Courses/212/}, 1992.
\end{thebibliography}

\end{document}
